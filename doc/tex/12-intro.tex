\Introduction
Анализ сетевых пакетов всегда был и остается актуальной темой. Связано это с бурно растущей популярностью сети Интернет и стремительным ростом ее пользователей. Ранее анализ пакетов использовался для обнаружения проблем в сети и их дальнейшего устранения, а также широко использовался злоумышленниками для получения конфиденциальных данных.

Большинство имеющихся на данный момент разработок позволяют перехватывать сетевой трафик, сохранять его в каком-то виде, предоставляют способы для работы с сохраненными данными. К таким программам можно отнести все сетевые снифферы (tcpdump, Wireshark).

Технология глубого анализа является относительно новым направлением в сетевой отрасли. Ключевая идея состоит в том, чтобы анализировать не только данные канального, сетевого и транспортного уровней, но и данные остальных уровней сетевой модели OSI. Рост вычислительных мощностей современных компьютеров позволяет проводить такой анализ с высокой производительностью, что делает эту технологию привлекательной для использования в реальных системах, например:
\begin{itemize}
\item для блокировки определенного сетевого трафика;
\item для организации транспорта в сети путем изменения содержимого пакетов опеределенного типа;
\item для сбора статистических данных по каждому типу интересующего трафика;
\end{itemize}

В реальных системах сетевой сервис классификации трафика на основе технологии глубокого анализа может использоваться для пересылки более приоритетных пакетов (например, голоса) по более мощным линиям связи, или, например, для блокирови торрентов. Положиться на данные сетевого и транспортного уровня на 100\% нельзя, так как их легко можно подменить.

Разрабатываемый продукт акселерирован Intel DPDK, что позволит получить высокие показатели производительности, а также же дает преимущество перед существующими аналогами.
