\Introduction
Анализ сетевых пакетов всегда был и остается актуальной темой. Связано это с быстро растущей популярностью сети Интернет и стремительным ростом числа пользователей. Ранее анализ пакетов использовался для обнаружения проблем в сети и их дальнейшего устранения, а также широко использовался злоумышленниками для получения конфиденциальных данных.

В сетях крупных интернет-провайдеров анализ содержимого пакетов является ключевым механизмом реализации таких функций, как:
\begin{itemize}
\item блокировка опеределенного типа трафика (полностью или частично);
\item реализация QoS;
\item балансировка нагрузки на каналах передачи данных;
\item организация маршрутов (например в MPLS сетях);
\end{itemize}

Обычно, все эти функции основаны на анализе mac-адресов, ip-адресов, портов транспортного уровня сетевой модели OSI, что лишает интернет-провайдеров таких возможностей, как, например, блокировка SIP трафика. Для этих целей может быть использована технология DPI.

Технология глубого анализа является, относительно, новым направлением в сетевой отрасли. Ключевая идея состоит в том, чтобы анализировать не только данные канального, сетевого и транспортного уровней, но и данные остальных уровней сетевой модели OSI.

Рост вычислительных мощностей современных компьютеров и использование фреймворков быстрой обработки пакетов, таких как DPDK, позволяют достичь высоких показателей производительности, что делает реальным использование чисто программных реализаций DPI в сетях интернет-провайдеров. Это позволит сэкономить средства, которые тратятся на закупку дорогостоящего оборудования.
