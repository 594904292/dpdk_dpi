\begin{abstract}
В данной дипломной работе рассмотрены вопросы проектирования и разработки сетевого сервиса классификации трафика на основе технологии DPI.

В аналитическом разделе подробно рассмотрена предметная область, проведен сравнительный анализ существующих аналогов разрабатываемого ПО и сформулированы требования к разрабатываемому сервису.

В конструкторском разделе приведены возможности и основные особенности фреймворка DPDK, который является ключевым звеном в разрабатываемом ПО. Также подробно описаны структура и детали использования протоколов, обнаружением которых будет заниматься разрабатываемый сервис. Разработаны алгоритмы обнаружения каждого из поддерживаемых протоколов. Описана структура и допустимые значения файла конфигурации.

В технологическом разделе описан выбор средств и технологии разработки, особенности, используемые при разработке данного ПО. Также описан процесс установки и запуска приложения, допустимые ключи и их значения.

В исследовательском разделе проведено экспериментальное исследование разработанного сетевого сервиса на тестовом наборе сетевых пакетов, а также сравнение по производительности с основными конкурентами.

В организационно-экономическом разделе приведены расчеты по определению структуры затрат на разработку проекта, а также выполнено планирование цены программного продукта и прогнозирование прибыли.

В разделе промышленной экологии и безопасности рассмотрены опасные и вредные факторы, влияющие на программиста при разработке, а также приведены расчеты уровня шума в серверной комнате.

В заключении делается вывод о результатах, достигнутых в ходе выполнения дипломной работы.
\end{abstract}
