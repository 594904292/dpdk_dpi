\Abbreviations
\begin{description}
\item[QoS] качество обслуживания (англ. Quality Of Service).
\item[FTP] протокол передачи файлов (англ. File Transfer Protocol).
\item[HTTP] протокол передачи гипертекста (англ. HyperText Transfer Protocol).
\item[SIP] протокол установления сессии (англ. Session Initiation Protocol).
\item[RTP] протокол передачи трафика реального времени (англ. Real time Transport Protocol).
\item[RTSP] потоковый протокол реального времени (англ. Real time Streaming Protocol).
\item[OSI] открытая модель взаимодействия систем (анг. Open Systems Interconnection).
\item[IP] межсетевой протокол (англ. Internet Protocol).
\item[TCP] протокол управления передачей (англ. Transmission Control Protocol).
\item[UDP] протокол пользовательских датаграмм (англ. User Datagram Protocol).
\item[DPI] глубокий анализ пакета (англ. Deep Packet Inspection).
\item[VLAN] виртуальная локальная сеть (англ. Virtual Local Area Network).
\item[MPLS] многопротокольная коммутация по меткам (англ. Multiprotocol Label Switching).
\item[FEC] класс сетевого уровня (англ. Forwarding Equivalence Class).
\item[LSR] маршрутизатор сети MPLS (англ. Label Switching Router).
\item[LSP] путь переключения меток (англ. Label Switching Path).
\item[LDP] протокол распределения меток (англ. Label Distribution Protocol).
\item[CPU] центральный процессор (анг. Central Processing Unit).
\item[TLB] буфер ассоциативной трансляции (англ. Translation Lookaside Buffer).
\item[DMA] прямой доступ к памяти (англ. Direct Memory Access).
\item[DCA] прямой доступ в кэш (англ. Direct Cache Access).
\item[RSS] распределение входящей нагрузки (англ. Receive Side Scalling).
\item[CRC] циклический избыточный код (англ. Cyclic Redundance Check).
\item[API] интерфейс программирования приложений (англ. Application Programming Interface).
\item[RFO] запрос на владение (англ. Request For Ownership).
\item[IDE] интегрированная среда разработки (англ. Integrated Development Environment).
\end{description}
